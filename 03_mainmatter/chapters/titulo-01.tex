\chapter{ORIGEN Y NATURALEZA DEL GRUPO}

\begin{itemize}
  \item [\texttt{Art 00 ::}] El presente estatuto se rige bajo las herramientas de la resolución rectorial Nº 0845.
  \item [\texttt{Art 01 ::}] La Asociación Estudiantil Four Leaves en siglas \textbf{FLAE}, es formado gracias a las iniciativas de un grupo de estudiantes de Matemática, Ing. Física y Ciencia de la Computación interesados en fomentar las matemáticas con relación a la computación. \textbf{FLAE} tiene como objetivo fundamental investigar y divulgar de manera simple la relación entre las Matemáticas y la Computación en el ambito académico y tecnológico.
  \item [\texttt{Art 02 ::}] Para cumplir su misíon, \textbf{FLAE} se basará en los siguientes principios:
  \begin{itemize}
    \item Libertad de conciencia.
    \item Nadie será discriminado por razón de sus ideas, credos políticos o religiosos ni por su posición social , económica o su orientación sexual.
  \end{itemize}
  \item [\texttt{Art 03 ::}] \textbf{FLAE}  tiene por visíon ser una asociación que fomente la interrelación entre académicos de todo el pais.
  \item [\texttt{Art 04 ::}] En concordancia con su visión \textbf{FLAE}  establece lo siguiente:
  \begin{itemize}
    \item Promover la divulgación mediante un contenido virtual, exposiciones o talleres.
    \item Promover el estudio e investigación de la Matemática, Física y Computación.

  \end{itemize}
\end{itemize}

% \section*{Capítulo 1}

% Los siguientes comandos de sección están disponibles:
% \begin{quote}
%  part \\
%  chapter \\% \\ Fuerza una nueva linea
%  section \\
%  subsection \\
%  subsubsection \\
%  paragraph \\
%  subparagraph
% \end{quote}% Fin del texto indentado.

% Pero tenga en cuenta que  a diferencia de las clases de \texttt{book} y \texttt{report}, la clase de \texttt{article} no tiene un comando \texttt{chapter}.


% Art 0 :: El presente estatuto se rige bajo las herramientas de la resolución rectorial Nº 0845.
%
% Art 1 :: La Asociación Estudiantil Four Leaves en siglas \textbf{FLAE}, es formado gracias a las iniciativas de un grupo de estudiantes de Matemática, Ing. Física y Ciencia de la Computación interesados en fomentar las matemáticas con relación a la computación. FLAE tiene como objetivo fundamental investigar y divulgar de manera simple la relación entre las Matemáticas y la Computación.
%
% Art 2 :: Para cumplir su misíon, FLAE se basará en los siguientes principios:
%   + Libertad de conciencia. Nadie será discriminado por razón de sus ideas, credos polı́ticos o religiosos ni por su posición social , económica o su orientación sexual.
%
% Art 3 :: FLAE tiene por visíon ser una asociación que fomente la interrelación entre académicos de todo el pais.
%
% Art 4 :: En concordancia con su visión FLAE establece lo siguiente:
%   + Promover el estudio e investigación de la Matemática, Física y Computación.
%   + Promover la divulgación mediante un contenido virtual, exposiciones o talleres.
