\chapter {ESTRUCTURA ORGÁNICA}
\section*{CAPÍTULO 1: DE LOS MIEMBROS EN GENERAL}
\begin{itemize}
  \item [\texttt{Art 05 ::}] Pueden ser considerados miembros de \texttt{FLAE}, las personas interesados en la Matemática, Ing. Física, Física y Ciencia de la Computación,  que realizan un compromiso y en sus actos muestran honrar ese compromiso. Quienes por propia voluntad desean apoyar al grupo sin
  realizar un compromiso podrán ser considerados como participantes auxiliares.
  \item [\texttt{Art 06 ::}] Los miembros se clasificarán  en tres categorías:
    \begin{itemize}
      \item Miembros Activos, quienes se comprometen en el grupo y por su disponibilidad de tiempo estén en la posibilidad de ejercer cargos administrativos.
      \item Miembros Asociados,quienes se comprometen con el grupo y a participar en la medida de sus posibilidades.
      \item Miembros Enbajadores , quienes se comprometen con el grupo en el apoyo divulgativo en otras facultades distintas a la facultad de ciencias UNI o en otras instituciones distintas.
    \end{itemize}
  \item [\texttt{Art 07 ::}] Todos los miembros de \texttt{FLAE} dentro de su denominación como miembros activos, miembros asociados o miembros embajadores, tienen los mismos derechos y obligaciones generales sin excepción.
  \item [\texttt{Art 08 ::}] Son derechos generales de los miembros:
  \begin{itemize}
    \item Ser comunicado oportunamente de las convocatiorias a las sesiones.
    \item Tener voz en las sesiones ordinarias y extraorinarias en las que asistan.
    \item Tener acceso  a la infrmación de los acuerdos tomados en las sesiones que no asistan.
  \end{itemize}
  \item [\texttt{Art 09 ::}] Son obligaciones generales de los miembros:
  \begin{itemize}
    \item Asistir a las sesiones ordinarias y extraordinarias ademas de actos académicos institucionales.
    \item Abonar las cuotas ordinarias y extraordinarias acordadas por el Grupo, de acuerdo con los principios.
    \item Cumplir las disposiciones del estatuto y de los acuerdos que se tomen conjuntamente.
  \end{itemize}
\end{itemize}

\section*{CAPÍTULO 2:  DEL RÉGIMEN ADMINISTRATIVO}
\begin{itemize}
  \item [\texttt{Art 10 ::}] \textbf{FLAE} en conjunto, reconoce la autoridad del Consejo de Facultad, acatando lo
  establecido en los diferentes dispositivos legales vigentes según lo dispuesto en el
  \textbf{Art.4} del Reglamento de Asociaciones Estudiantiles de la Facultad de Ciencias
  y de la Resolución Rectoral Nº 0845.
  \item [\texttt{Art 11 ::}] Son entes de la organización insitutucional dentro \flae:
  \begin{itemize}
    \item La asamblea general.
    \item La junta directiva.
  \end{itemize}
  \item [\texttt{Art 12 ::}] La Asamblea General la integran todos los miembros de \flae, se reúnen en
  Sesión Ordinaria por lo menos una vez al año para renovar el compromiso de los miembros, evaluar el desempeño de la Junta Directiva a cargo y elegir la nueva Junta Directiva.
  \item [\texttt{Art 13 ::}] La Junta Directiva es la autoridad representativa del  \flae, la elige la Asamblea
  General en votación por mayoría y se renueva por lo menos una vez al año.
  \item [\texttt{Art 14 ::}] \flae  contará con la supervisión del por los menos un profesor asesor. La Junta
  Directiva propondrá al Consejo de Facultad el nombre del profesor asesor, para su confirmación.
\end{itemize}


\section{CAPÍTULO III: DE LA JUNTA DIRECTIVA}
\begin{itemize}
  \item [\texttt{Art 15 ::}] La Junta Directiva es el organismo ejecutivo de \flae y está constituida por los
  miembros activos que desempeñan los siguientes cargos:
  \begin{itemize}
    \item Presidente
    \item Vicepresidente
    \item Secretario
  \end{itemize}
  \item [\texttt{Art 16 ::}] Son atribuciones de la Junta Directiva:
  \begin{itemize}
    \item Gestionar y cumplir los acuerdos tomados en las sesiones.
    \item Ejercer la representación del FLAE ante las autoridades universitarias y otras instituciones.
    \item Informar a sus miembros sobre toda gestión que realice.
    \item Entrevistar a los candidatos a miembros.
  \end{itemize}
  \item [\texttt{Art 17 ::}]  Son atribuciones del Presidente:
  \begin{itemize}
    \item Asistir a las reuniones de \flae.
    \item Validar con su firma los documentos de obligación y trámite administrativos y académicos.
    \item Hacer cumplir los acuerdos adoptados en las sesiones.
  \end{itemize}
  \item [\texttt{Art 18 ::}] Son atribuciones del Vicepresidente:
  \begin{itemize}
    \item Realizar las atribuciones del Presidente cuando éste se encuentra ausente.
    \item Asistir a las sesiones en ausencia del presidente.
  \end{itemize}
  \item [\texttt{Art 19 ::}] Son atribuciones del Secretario:
  \begin{itemize}
    \item Redactar los comunicados y acuerdos tomados en las sesiones de \flae.
    \item Realizar los trámites de funciones administrativas correspondientes a su cargo.
    \item Llevar el control de los trabajos que realizan los miembros de \flae.
    \item Presentar un informe al final de su ejercicio, dando cuenta de la labor realizada por la Junta Directiva, bajo su responsabilidad.
  \end{itemize}

\end{itemize}


\section*{CAPÍTULO 4: DE LOS MIEMBROS ACTIVOS}
\begin{itemize}
  \item [\texttt{Art 20 ::}] Pueden ser miembros activos los alumnos matriculados en la Universidad Nacional
  de Ingeniería, que cuenten con un promedio aprobatorio de los dos últimos semestres
  académicos cursados y que por propia iniciativa decidan adherirse realizando un compromiso.
  \item [\texttt{Art 21 ::}] Son requisitos para ser Miembro Activo:
  \begin{itemize}
    \item No encontrarse sometido a proceso disciplinario.
    \item Tener promedio aprobatorio de los dos últimos ciclos cursados.
    \item Haber pasado el proceso de admisión de miembros.
    \item Disponer de tiempo adecuado para la realización de sus actividades como Miembro Activo.

  \end{itemize}
  \item [\texttt{Art 22 ::}] Son derechos de los Miembros Activos:
  \begin{itemize}
    \item Postular a algún cargo de la Junta Directiva.
    \item Participar activamente en cualquiera de las actividades o proyecto de FLAE.
    \item Solicitar hasta un crédito por actividades extracurriculares, de acuerdo al Art. 17 de la Resolución Rectoral Nº 0845.
  \end{itemize}
  \item [\texttt{Art 23 ::}] Son obligaciones de los Miembros Activos:
  \begin{itemize}
    \item Disponer del tiempo suficiente para realizar todas las actividades que como Miembro Activo le correspondan. Dichas actividades serán designadas por la Junta Directiva.
    \item La autoformación en los campos en los que se desarrolla dentro del grupo, así como promover el interés por desarrollar más campos de aplicación según los fines del grupo.
    \item Estar comprometidos con la misión y visión de \flae y mantener acuerdo de confidencialidad sobre temas de interés grupal, salvo que el Consejo de Facultad solicite la información.
  \end{itemize}

\end{itemize}

\section*{CAPÍTULO 5: DE LOS MIEMBROS ASOCIADOS}
\begin{itemize}
  \item [\texttt{Art 24 ::}] Pueden ser miembros asociados los alumnos matriculados o egresados de la UNI,
  que por propia iniciativa decidan adherirse realizando un compromiso.
  \item [\texttt{Art 25 ::}]  Son requisitos para ser Miembro Asociado:
  \begin{itemize}
    \item Haber sido considerado participante auxiliar por un tiempo por la Junta Directiva.
    \item Ser nombrado como tal por acuerdo de los miembros en Sesión Ordinaria.
    \item Demostrar su interés por al menos una de las actividades ejercidas por \flae.

  \end{itemize}
  \item [\texttt{Art 26 ::}] Son obligaciones de los Miembros Asociados:
  \begin{itemize}
    \item Disponer del tiempo suficiente para realizar las actividades que como Miembro Asociado le correspondan. Dichas actividades serán designadas por la Junta Directiva.
    \item Mantenerse informado de las actividades que realiza \flae.
  \end{itemize}

\end{itemize}

\section*{CAPÍTULO 6: DE LOS MIEMBROS EMBAJADORES}
\begin{itemize}
  \item [\texttt{Art 27 ::}] Pueden ser miembros embajadores los alumnos matriculados o egresados de diversas instituciones.
  \item [\texttt{Art 28 ::}] Son requisitos para ser Miembro Embajador:
  \begin{itemize}
    \item Ser nombrado como tal por acuerdo de los miembros en sesión ordinaria.
    \item Demostrar su interés por la divulgación de las actividades de \flae en su institución.
  \end{itemize}
  \item [\texttt{Art 29 ::}] Son derechos de los Miembros Embajadores:
  \begin{itemize}
    \item Tener voz mas no voto en las sesiones.
    \item Participar en la medida de sus posibilidades en las actividades programadas por la asociación.
  \end{itemize}
  \item [\texttt{Art 30 ::}] Son obligaciones de los Miembros Embajadores:
  \begin{itemize}
    \item Disponer del tiempo suficiente para realizar las actividades que como Miembro Embajador le correspondan. Dichas actividades serán designadas por la Junta Directiva.
    \item Mantenerse informado de las actividades que realiza \flae.
  \end{itemize}

\end{itemize}
