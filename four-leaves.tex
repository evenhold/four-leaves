% arara: clean: {extensions: ['log','aux','fdb_latexmk','fls','ilg','synctex.gz','toc','out']}
% arara: xelatex
% arara: clean: {extensions: ['log','aux','fdb_latexmk','fls','ilg','synctex.gz','toc','out']}
\documentclass[11pt, oneside]{book}
\usepackage[colorlinks,citecolor=blue,linkcolor=red]{hyperref}
\usepackage{lipsum}

\renewcommand{\contentsname}{Contenido}

\begin{document}

% ===== Título de portada ====
\begin{titlepage}
\raggedleft % Alinear todo a la derecha
\vspace*{\baselineskip} % Espacios en blanco en la parte superior de la página

{\Large Franz Victorio \\ Carlos Aznarán \\ Freider Achic \\ Renzo Q. Amao \\ } \vspace*{0.167\textheight}


\textbf{\LARGE  Guía de inicio }\\[\baselineskip]

{{\Huge \bf Four Leaves}}\\[\baselineskip]

{\large \bf Organización}

\vfill %Espacio en blanco entre los títulos y el editor.

{\large\bf Evenhold }
\vspace{3\baselineskip} % La versión * solo va cuando no hay texto en la hoja donde se edita.

\end{titlepage}
% ======================= %
\tableofcontents

\chapter*{Prefacio}
\lipsum[1]

\chapter*{Prólogo}
\lipsum[1]

\chapter{¿Qué es GNU/Linux?}
\lipsum[1] \texttt{article} and \texttt{book} classes as well as to \texttt{report} class. In \texttt{article} class, however, the default position for the title information is at the top of the first text page rather than on a separate page. Also, it is not usual to request a table of contents with \texttt{article} class.

\section{Subtítulo}

Los siguientes comandos de sección están disponibles:
\begin{quote}
 part \\
 chapter \\% \\ Fuerza una nueva linea
 section \\
 subsection \\
 subsubsection \\
 paragraph \\
 subparagraph
\end{quote}% Fin del texto indentado.

Pero tenga en cuenta que  a diferencia de las clases de \texttt{book} y \texttt{report}, la clase de \texttt{article} no tiene un comando \texttt{chapter}.

\chapter{Comandos básicos en GNU/Linux}
\lipsum[1]

\end{document}
